\chapter{Conclusion}
\label{chap:conclusion}

\section {Summary}
In this thesis, we have provided an abstraction and implementation of a general purpose text indexing system in Haskell.
We gave a brief introduction to functional programming and its related concepts in Haskell,
followed by text indexing in general and then went on to describe how using type safety and advanced concurrency models in Haskell
we implemented the text indexing engine.

The design is inspired by real world tested and proven frameworks like Lucene combined with concepts taken from other systems like maildir format\cite{bernsteinMaildir}
and we have also complemented our implementation with comprehensive documentation and test suites.

\section{Future Work}
Currently the implementation uses some POSIX specific functionality and thus would not run on Windows out of the box.
However in principle there is no hard dependency on any platform and its just a matter of porting the POSIX system calls to Windows Userland.

Also in order for this to be usable directly in any application a lot of functionality needs to be built on top of this implementation.
For example before indexing the content must be pre-processed to remove stop words, terms must be stemmed, etc.
A general overview of the kind of functionality that is generally required is provided in Chapter \ref{chap:textIndexing}

On the other hand, this project is a part of a larger project to provide a key-value and binary data store along with a text indexing framework.
Efforts are underway to integrate all these projects into a full fledged solution.

%%% Local Variables:
%%% mode: latex
%%% TeX-master: "thesis"
%%% End:
