\chapter{Text Indexing}
\label{chap:textIndexing}

``Information retrieval (IR) is finding material (usually documents) of an unstructured nature (usually text)
that satisfies an information need from within large collections (usually stored on computers).''\cite{IRBook}

Full text search forms an integral part of Information Retrieval Systems which aim at extracting
relevant documents from a collection based on certain keywords.
It differs from other methods which are based on metadata or other parts of documents searching as it exclusively focuses on the
textual content of the document and unlike exact matches or range based searches has to interpret unstructured data.

At this point it is worth mentioning that this thesis does not try to provide an end to end text indexing solution.
Rather it tries to provide a framework for building such solutions borrowing concepts from IR literature and actual software
implementations to provide a generic yet powerful abstraction.
At the same time it also assumes a certain structure to the expected interface which reflects in its implementation.
However the two might be viewed independently if necessary and the aim is to make the aforementioned implementation a specific
case of the abstraction which in its own would hold for any generic text retrieval system.

This chapter aims to serve as a text indexing primer to the reader.
The terms introduced here would form the basis of nomenclature in the rest of this document and
the concepts would help the reader understand the aspirations behind the design and appreciate the why, what and how
of the implementation details.

\section{Index}
The standard information retrieval problem can be simplified as follows:

\noindent \textbf{Given a list of documents and a search term, return all the documents which contain the term}

One way to do that would be to iterate over all the documents, scan them from beginning to end and return those with the search term in it.
This is commonly known as \textit{grepping}, after the UNIX utility \textit{grep}.
Though this process requires linear time (linear in the number of documents),
it is sufficient in cases where the number of documents times the length of such documents is small enough.

On the other hand, web search, in which a term is to matched against all known web pages, is a huge tasking that cannot be computed online as before.
Thus there is a need for an efficient mechanism to retrieve relevant documents from a large set in a reasonable amount of time.
To overcome the need for scanning all the documents every time a query is performed we build an \textit{index} of terms corresponding to
documents they occur in thus reducing the problem to looking up the term in this \textit{index} instead of looking up the entire document set.

\subsection{Incidence Matrix}
An Incidence Matrix is a binary term-document mapping where each entry describes whether the term is present in the document or not.
For example a list of collaborators of Martin Scorsese in his movies, the incidence matrix would be something like

\begin{table}[ht]
  \caption{Incidence Matrix of Martin Scorsese's Collaborators}
  \centering
  \begin{tabular}{p{4cm}p{2cm}p{2cm}p{2cm}p{2cm}}
    \hline \hline
                      & Taxi Driver & Gangs of New York & Shutter Island & Wolf of Wall Street  \\
    \hline
    Robert De Niro    &      ✓      &                   &                &                      \\
    Leonardo DiCaprio &             &        ✓          &         ✓      &           ✓          \\
    Daniel Day Lewis  &             &        ✓          &                &                      \\
    Jodie Foster      &      ✓      &                   &                &                      \\
  \end{tabular}
\end{table}

One may observe that this requires $O(n*m)$ space where $n$ is the number of terms and $m$ is the number of documents.
Moreover this is generally a sparse matrix.

\subsection{Inverted Index}
Because the incidence matrix is generally a sparse matrix, it is preferable to store only the \textbf{✓}s
i.e. those entries where the term occurs in the document.
\begin{table}[ht]
  \caption{Inverted Index of Martin Scorsese's Collaborators}
  \centering
  \begin{tabular}{p{4cm}p{7cm}}
    \hline \hline
    Robert De Niro    & [Taxi Driver] \\
    Leonardo DiCaprio & [Gangs of New York, Shutter Island, Wolf of Wall Street] \\
    Daniel Day Lewis  & [Gangs of New York] \\
    Jodie Foster      & [Taxi Driver] \\
  \end{tabular}
\end{table}

This matrix is represented as a dictionary of \textit{terms} against a list of records which points to the list of documents in which it occurs.
This record is called a \textit{posting} and generally besides the document identifier it also contains some meta information about the
occurrence of a term in the document (for example the term frequency, term vectors, etc.).
The list of postings corresponding to a term is referred to as \textit{postings list} which is stored in order of document ids
and the collection of such a mapping from \textit{term} to \textit{postings list} is called the \textit{dictionary}.

\section{Building the Index}

\subsection{Extracting Terms}
\label{terms}
Before the index can be built, the list of terms need to be extracted from the document. This in turn is composed of multiple tasks like
\begin{itemize}
  \item \textbf{Tokenization} is the process of extracting semantically meaningful identifiers from the document.
    It is language dependent and can range from simple implementations like delimiter splitting the text to advanced and more complex
    methods involving language detection, hyphenation sensitive parsing, compound word handling, etc.
  \item \textbf{Stop word elimination} involves extremely commonly used words like pronouns, conjunctions, etc.\ from the tokens which do
    not add value to the retrieval process.
  \item \textbf{Normalization} refers to the process of canonicalizing terms such that semantically similar terms are grouped under
    same equivalence classes. This ensures that the search process can adjust for case differences, accent and diacritics and other
    natural language specific issues.
  \item \textbf{Stemming and Lemmatization} Stemming reduces inflected words to their stem word (the base word in simpler terms) whereas
    Lemmatization achieves the same goal through morphological analysis of words.
\end{itemize}
Extracting terms from a document might include all or some of these processes.
The end result of this process results in the list of terms corresponding to that document which forms the base for building the index.


\subsection{Postings}
\label{postings}
The associated meta information in the mapping of terms to document ids are referred to as the \textit{postings list}
The underlying data structure plays an important role in the efficiency of the search process.
The most important operations on the posting list are \textit{union}, \textit{intersection} and \textit{difference},
thus it is important that the underlying representation allows us to perform these operations efficiently.

A simple representation using unordered list offers a time bound of $O(m*log n)$ where $m$ and $n$ are the lengths of the list respectively.
However by simply ordering the list we can achieve an efficiency of $O(m + n)$

Generally simple ordered lists of single terms are not enough for some advanced use cases.
For example phrase searches require the postings format to be flexible enough to support such queries.
The exhaustive details of all such possible operations is beyond the scope of the project but we will see how this can be represented
in \textit{Haskell} effectively.


\subsection{Dictionary}
\label{dictionary}
\textbf{Dictionaries} refer to the data structure used to store the mapping from \textit{term}s to \textit{postings}.
Generally the dictionary is stored in memory with a reference to the postings list on disk.
The dictionary format plays an important role in determining the type of queries that can be performed on the index.
For example \textit{wildcard queries} and imprecise query search like \textit{spelling errors} are very common use cases in any text retrieval system.
Trees are generally the data structures of choice in most applications.
It allows the user to perform various kinds of traversals which in turn add to the power of search operations.

\textbf{Trie} has been our choice of data structure (for reasons explained later in this document)


\section{Querying the Index}
Querying the index generally involves two steps.
First looking up the dictionary to retrieve the pointer (offset in the file) to the document.
And then looking up the document according to the pointer.
However just retrieving the documents isn't enough. The documents need to be sorted according to relevance.
For this we need a measure of relevance of the documents, which might depend on the metadata of the document or the content or both.
Different models have been proposed for measuring the relevance of the query results based on term frequencies, term vectors, etc.
The model of choice is generally application specific.

%%% Local Variables:
%%% mode: latex
%%% TeX-master: "thesis"
%%% End:
