\chapter{Functional Programming}
\label{chap:FuncProg}

Functional Programming is a programming paradigm which treats functions as first class members
and computations are modeled as the evaluation of mathematical functions i.e. the output of functions
depends only on the input parameters and not on a globally modifiable state or data.
This paradigm allows us to better argue about the semantics and the correctness of the program as the output of each function,
which is the building block of the program, depends only on the the set of input provided and through
mathematical constructs the correctness of the program can be expressed as the composition of the correctness measures
of the constituent functions.
Functional programming was born out of Lambda Calculus, a formal system described by Alonzo Church, to study computability.

\section{Purely Functional}
In contrast to imperative languages which view computation as global state and a set of commands modifying this shared state,
declarative programming proposes an alternate programming model in which the emphasis is on the logic of the program rather
than the control flow.
In this respect, \textbf{referential transparency} is a key concept which means that an expression can always be replaced with its
value and that this replacement should not change the result of the program.
Referential transparency aids in arguments about the correctness and behavior of the program, can greatly help in optimizations
like memoization, code transformations, converting to lazy or strict evaluation and parallelisation optimizations that are
otherwise difficult in imperative languages.

\section{Higher Order Functions}
In functional languages like Haskell functions are treated as first class variables meaning they can be passed to functions
and functions can return functions just like any variable. This combined with the concept of partial application of arguments to a
function know as \textbf{currying} provides a powerful programming construct.

An example of this would be the function \texttt{map} which takes a function and a list of arguments and applies the function to the list
generating a new list. Using this and applying it on a function that takes an integer and increments it by \texttt{1} would give us another
function that takes a list of integers and increments all of them by \texttt{1}.

\begin{listing}
\inputminted{haskell}{hs/higher_order.hs}
\caption{Higher Order Functions}
\end{listing}

\section{Powerful Type System}
Haskell provides a powerful type system in which each value has an associated type.
Type system enables program verification by imposing constraints on value and aiding in systematic checking of those systems using mathematically
sound and complete proving systems.
Haskell provides a strongly and statically checked type system along with powerful constructs to express higher level and more expressive types.
While \textbf{type inference} relieves the user from explicitly having to define types with each individual value,
the presence of \textbf{abstract data types} and \textbf{pattern matching} makes the use of complex data types convenient.

\subsection{Static Typing}
Haskell is a statically typed language meaning that all type checks happen at compile time. This is in contrast to dynamically typed languages
in which all type checking happens at run-time. The benefits of static type checking is in the fact that the compiler can test and verify the
program before it is actually executed, thus preventing from real world crashes that could have occurred otherwise.
Also compile time type checking means that the type information can be erased at run-time resulting in no run-time overhead of checking
and ensuring types.

\subsection{Strongly Typed}
Strongly typed means that every value must be associated with a type.
Liskov and Zilles define strong typing as ``whenever an object is passed from a calling function to a called function,
 its type must be compatible with the type declared in the called function''\cite{LZ74}.
Generally strongly typed languages disallow direct type casting of one type to another and implicit casts.

\subsection{Pattern Matching}
Abstract data types along with pattern matching provide a new way to structure programs.
Abstract data types can capture mathematical structures as they appear including recursive structures and multi-value types.

Pattern matching on the other hand helps break down the compound structure into constituent units.
They are like guarded expressions which help break the function into sub-functions depending on the type or value of variables involved.
\begin{listing}
\inputminted{haskell}{hs/fibonacci1.hs}
\caption{Computing Fibonacci using Pattern matching}
\end{listing}


\section{Lazy Evaluation}
Lazy evaluation is a the concept of \textbf{call by need} in which the value of an expression is only calculated when it is required.
The benefits of lazy evaluation include
\begin{packed_itemize}
\item Performance benefits by identifying redundant or repeating computations and performing optimizations
\item Creation and operations on infinite data structures
\item Allows for complex control flow structures like value recursion, in which a value is defined in terms of itself
\end{packed_itemize}

\begin{listing}
\inputminted{haskell}{hs/fibonacci.hs}
\caption{Computing Fibonacci using Lazy Evaluation}
\end{listing}

\section{Concurrency Primitives}
Concurrent Haskell\cite{jones1996concurrent} extended the semantics of Haskell to include concurrency alongside lazy functional code.
Concurrency in Haskell can be described using three main concepts
\begin{description}
\item[Mutable Var] A mutable variable can be thought of as a reference to a mutable location in memory which can either be empty or hold
a value of type \textbf{$\alpha$}.
\item[Concurrent Threads] Concurrent Haskell provided a new primitive \texttt{forkIO} which starts a concurrent process.
\item[Software Transactional Memory] Software Transactional Memory or their implementation \textbf{STM}\cite{harris2005composable}
is a way to model concurrent programs using shared memory without locks. The idea is similar to database transactions in which multiple
reads and writes can happen to shared memory such that intermediate states are not visible to other transactions.
\end{description}

%\section{Extensions, Polymorphism \& Beyond}

%%% Local Variables:
%%% mode: latex
%%% TeX-master: "thesis"
%%% End:
