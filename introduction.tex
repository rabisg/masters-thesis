\chapter{Introduction}
\label{chap:intro}
Persistence and retrieval of stored data forms an integral part of any application.
A database management system (DBMS) is a specialized software build for organizing large collection of data
whereas a database index (or just index) is a data structure, often build into the DBMS,
that provides an efficient and fast way to query the collection without having to go through the entire dataset.

Over the years the approach to databases have evolved from navigational databases
to relational model, first proposed by Edgar F. Codd\cite{codd1970relational},
and more recently to fast key-value and document based stores.
These post-relational databases came to be known as NoSQL, a term coined by Carlo Strozzi\cite{strozziNoSQL},
and have been widely used in big data and real-time applications.

Indexing has also been a widely studied and researched topic as a part of Information Retrieval Systems.
The data structures required for indexing depend highly on the use case.
For example, \textit{PostgreSQL} which is one of the most commonly used SQL databases has four major types of indexes.
Each index type is based on a different set of algorithms and data structures which provides a specific set of
functionality where functionality refers to the kind of operations (operators) that index type can support.
The most commonly implemented index type is a \textbf{B-Tree} based index which supports equality and range based
operations and provides good performance for commonly used SQL types like numeric data, strings and dates.
However a \textbf{B-Tree} or \textbf{B*-Tree} based index is not sufficient for all types of access patterns.
Specifically text based retrieval in which a word or a phrase is to be matched against a given set of documents is one
such application where \textbf{B*-Tree} based indexes is generally not the data structure of choice.
Text retrieval, as it is called, has been an area of active research and development.
Different data structures and algorithms have been proposed which not only focus on the correctness of the results
but also on their relevance.

\section{Motivation}
For relevance based matching text-search engines need to store meta-data besides the actual document identifiers.
For simple text-search where only correctness matters storing the document reference is enough.
In all cases, the building block of any text based retrieval system is having a text to value mapping where
value can range from simple document identifiers to compound data structures storing meta information like the frequency
of the term or score of the document (based on some scoring mechanism).
Also the keys, which are arrays of characters in this case, need to support advanced operations in order for
the search engine to provide additional facilities (for example approximate or fuzzy search).
Thus a general purpose indexing based optimized key-value store can serve as the building block in a lot of applications.

\section{Related Work}

\subsection{Apache Lucene}
Lucene, a free and open source library for indexing and searching, was originally written by Doug Cutting\cite{goetz2000lucene}
in Java. Doug originally wrote it in 1999 but later it became a part of Apache Software Foundation under its Jakarta family
of open source Java projects.
Lucene and its derivatives are widely used and it has been known for its powerful, accurate and efficient search algorithms.
Over the years many developers have contributed resulting in advanced features like suggester, highlighting support, flexible faceting, etc.
ElasticSearch and Apache Solr are two examples of application build on top of this library that provide enterprise class search
platform with features like replication, rich document support, database integration and some NoSQL features.

\subsection{Xapian}
Xapian is also a free and open source library supporting full text search and probabilistic matching.
It is written in C++ and provides features like Transaction Support, spelling correction, synonym support, etc.
Omega is a packaged solution on top of Xapian which provides a search engine like support to other applications.

\section{Outline}
Chapter~\ref{chap:FuncProg} describes the aspects of functional programming and
Chapter~\ref{chap:textIndexing} provides the details of a generic text indexing system.
This chapter also provides the basis for understanding the design.
Chapter~\ref{chap:implementation} deals with the implementation details and design for the system.
Finally Chapter~\ref{chap:conclusion} concludes the thesis and describes the future work to be done in this field.

%%% Local Variables:
%%% mode: latex
%%% TeX-master: "thesis"
%%% End:
