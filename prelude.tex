% prelude.tex
%   - titlepage
%   - dedication (optional)
%   - approval sheet
%   - course certificate
%   - table of contents, list of tables and list of figures
%   - nomenclature
%   - abstract
%============================================================================


\clearpage\pagenumbering{roman}  % This makes the page numbers Roman (i, ii, etc)



% TITLE PAGE
%   - define \title{} \author{} \date{}
\title{TA × I - Text aggregating and indexing library for Haskell}
\author{Rabi Shanker Guha}
\date{June, 2015}

%  - Roll number, required for title page, approval sheet, and
%    certificate of course work
\rollnum{10327551}

%   - The default degree is ``Doctor of Philosophy''
%     (unless the document style msthesis is specified
%      and then the default degree is ``Master of Science'')
%     Degree can be changed using the command \iitbdegree{}
\iitbdegree{Master of Technology}

%   - The default report type is preliminary report.
%      * for a PhD thesis, specify \thesis
\thesis
%      * for a M.Tech./M.Phil./M.Des./M.S. dissertation, specify \dissertation
%\dissertation
%      * for a DIIT/B.Tech./M.Sc.project report, specify \project
%\project
%      * for any other type, use  \reporttype{}
%\reporttype{ReportType}

%   - The default department is ``Unknown Department''
%     The department can be changed using the command \department{}
\department{Computer Science \& Engineering}

%    - Set the guide's name
\setguide{Piyush P. Kurur}
\setguidedept{Department of Computer Science \& Engineering}

%   - once the above are defined, use \maketitle to generate the titlepage
\maketitle

%--------------------------------------------------------------------%
% CERTIFICATE
%     The first page after the title page.
\makecertificate

%--------------------------------------------------------------------%
% COPYRIGHT PAGE
%   - To include a copyright page use \copyrightpage
% \copyrightpage

%--------------------------------------------------------------------%
% ABSTRACT
\begin{abstract}
In this thesis we use Haskell, a polymorphic strongly statically typed, purely functional language to build a text indexing framework.
We build upon the concepts used in frameworks like Lucene and use Haskell's advanced type safety and concurrency features to
provide a generic framework for the indexing system.
The system is designed to be concurrent and operations to be thread safe while segments help
make the system easy to parallelize for future extension in a distributed environment.
We also explore various data structures for dictionary representation and argue why a Trie based implementation
based on multilevel Patricia Trees representation is our data structure of choice.
\end{abstract}

%--------------------------------------------------------------------%
% DEDICATION
%   Dedications, if any.
\begin{dedication}
To Incredible !ndia
\end{dedication}

% Acknowledgements
\begin{acknowledgments}
\end{acknowledgments}

%--------------------------------------------------------------------%
% CONTENTS, TABLES, FIGURES
\tableofcontents
\listoftables

\cleardoublepage
%\phantomsection \label{listoffig}
\addcontentsline{toc}{chapter}{List of Figures}
%\listof{program}{List of Figures}

\renewcommand\listingscaption{Program}
\renewcommand\listoflistingscaption{List of Programs}
\listoflistings

\listoffigures{}

\cleardoublepage
\pagenumbering{arabic} % Make the page numbers Arabic (1, 2, etc)

%%% Local Variables:
%%% mode: latex
%%% TeX-master: "thesis"
%%% End:
